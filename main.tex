\documentclass[zihao = -4,cn]{oucart}
\newtheorem{theorem}{{定理}}[section]
\newtheorem{proposition}{{命题}}[section]
\newtheorem{lemma}{{引理}} [section]
\newtheorem{corollary}{{推论}}[section]
\newtheorem{definition}{{定义}} [section]
\newtheorem{example}{{例}} [section]

\title{\textcolor[rgb]{0,0.19607843,0.7372549}\LaTeX{}模版使用手册}
\entitle{Handbook of OUC Undergraduate Thesis\\ \textcolor[rgb]{0,0.19607843,0.7372549}\LaTeX{}  Template}
\author{作者名}
\studentid{123456789}
\advisor{指导教师名}
\department{学院名}{专业年级}

\cnabstractkeywords{
 此项目旨在建立一个简单易用的中国海洋大学本科毕业论文\LaTeX{}模版。
}{
中国海洋大学,\LaTeX{}
}

\begin{document}

\makecover

\thispagestyle{empty}
\vspace{40mm}
\begin{center}
\heiti\zihao{3}{免责声明}
\end{center}

\begin{enumerate}
\item 本模板的发布遵守The MIT License,使用前请认真阅读协议内容。
\item 本模板为作者根据中国海洋大学教务处颁发的《中国海洋大学全日制本科毕业(设计)论文撰写规范》编写而成, 旨在供中国海洋大学本科毕业生撰写学位论文使用。
\item 中国海洋大学教务处只提供毕业论文写作指南,不提供官方模板,也不会授权第三方模板为官方模板,所以此模板仅为写作指南的参考实现,不保证格式审查老师不提意见。任何由 于使用本模板而引起的论文格式审查问题均与本模板作者无关。
\item 任何个人或组织以本模板为基础进行修改、扩展而生成的新的专用模板,请严格遵守协议The MIT License。由于违犯协议而引起的任何纠纷争端均与本模板作者无关。
\end{enumerate}
\newpage

\makeabstract

\thispagestyle{tableofcontents}  
\tableofcontents

\newpage
\pagenumbering{arabic}
\setcounter{page}{1} 
% 正文内容
% 建议使用 \input{<文件名>} 指令引用其他文件
\section{模版介绍}
此项目是基于\LaTeX{}排版系统的本科毕业论文模版,旨在帮助中国海洋大学的本科生快速完成毕业论文的排版工作。
\section{安装}
请根据你的操作系统选择对应的安装说明:
\subsection{Windows}
\begin{enumerate}
	\item 安装MiKTeX最新版本,下载地址与安装方法:\href{https://miktex.org/howto/install-miktex}{MiKTeX};
	\item 下载毕业论文论文模版的最新发行版,下载地址:\href{https://github.com/OSOUC/UndergraduateThesisLaTeXTemplate/releases}{GitHub};
	\item 在项目所在目录下编译生成PDF文件:
	\begin{itemize}
		\item[*]  自动编译:make\footnote{需要安装MinGW编译器。}
		\item[*]  手动编译:xelatex -> bibtex -> xelatex -> xelatex
	\end{itemize}
\end{enumerate}
\subsection{macOS}
\begin{enumerate}
	\item 安装MacTeX,下载地址:\href{http://www.tug.org/mactex/}{MacTeX};
	\item 下载毕业论文论文模版的最新发行版,下载地址:\href{https://github.com/OSOUC/UndergraduateThesisLaTeXTemplate/releases}{GitHub};
	\item 修改项目文件:\ttfamily{oucart.cls},\songti
	将以下代码:
	\begin{lstlisting}
\setCJKmainfont{SimSun}
\setCJKsansfont{SimHei}
\setCJKmonofont{FangSong}
	\end{lstlisting}\zihao{-4}
	更改为:
		\begin{lstlisting}
\setCJKmainfont[BoldFont=STHeiti,ItalicFont=STKaiti]{STSong}
\setCJKsansfont[BoldFont=STHeiti]{STXihei}
\setCJKmonofont{STFangsong}
	\end{lstlisting}\zihao{-4} 
	\item 在项目所在目录下编译生成\rmfamily{PDF} \songti 文件:
	\begin{itemize}
		\item[*]  自动编译:make
		\item[*]  手动编译:xelatex -> bibtex -> xelatex -> xelatex
	\end{itemize}
\end{enumerate}
\subsection{Linux}
待完成
\section{使用说明}
\subsection{示例文件}
模版包括封面、扉页、目录、中文摘要、英文摘要、正文、参考文献、致谢和附录,共9部分。以下是生成论文全部部分的示例代码,在论文的实际写作中只需要替换相应的部分即可:
\begin{lstlisting}
\documentclass[zihao = -4,cn]{oucart}
\title{论文标题}
\entitle{The Title of the Thesis}
\author{作者名}
\studentid{123456789}
\advisor{指导教师名}
\department{学院名}{专业年级}
\cnabstractkeywords{
 这是中文摘要。 
}{
中文, 关键字
}
\enabstractkeywords{
  This is an English abstract. 
}{
  English, Abstract
}
\begin{document}
\makecover
\makesignature
\makeabstract
\thispagestyle{tableofcontents}  
\tableofcontents
\newpage
\pagenumbering{arabic}
\setcounter{page}{1} 
% 正文内容
% 建议使用 \input{<文件名>} 指令引用其他文件
\section{示例章节}
\subsection{示例章节}
\subsubsection{示例章节}
正文\cite{wiki:ouc}. 
\newpage
\bibliography{main}

\newpage
\begin{center}
\zihao{3} \textbf{致谢} \\
\end{center}

\newpage
\begin{center}
\zihao{3} \textbf{附录} \\
\end{center}
\end{document}
\end{lstlisting}
\subsection{格式设置} \zihao{-4}
此模版按本科毕业论文格式规范制作,一般无需修改。若有模版格式与要求不一致的地方请根据以下说明进行修改:
\subsubsection{页面设置}
默认纸面为A4,页边距为上下2.54厘米,左右3厘米。页面设置在\ttfamily{oucart.cls} \songti 文件中,相关代码为:
\begin{itemize}
\item[*] 页面设置:
\begin{lstlisting}
\LoadClass[11pt,a4paper]{article}
\end{lstlisting}
\item[*] 页边距设置:
\begin{lstlisting}
\geometry{top=2.54cm,bottom=2.54cm,left=3cm,right=3cm} 
\end{lstlisting}
\end{itemize}
\subsubsection{页眉、页脚}
论文封面、扉页无页眉,中文摘要页页眉内容为摘要,英文摘要页页眉内容为Abstract,目录页页眉内容为目录,正文页页眉内容为论文题目,宋体居中。页码居中,正文页码为阿拉伯数字,摘要至目录页页码为大写罗马数字。页眉页脚的设置依赖\ttfamily{fancyhdr} \rmfamily \songti 宏包。正文以前部分的页眉样式设置代码如下:
\begin{lstlisting}
\fancypagestyle{zhabstract}{
\chead{\small{摘\ \ \ 要}}
}
\fancypagestyle{enabstract}{
\chead{\small{Abstract}}
}
\fancypagestyle{tableofcontents}{
\chead{\small{目\ \ \ 录}}
}
\end{lstlisting}
正文页眉设置为:
\begin{lstlisting}
\chead{\small{\@title}} % 页眉
\end{lstlisting}
以下代码用以设置页脚:
\begin{lstlisting}
\pagenumbering{Roman}  %大写罗马数字
\pagenumbering{arabic}    %阿拉伯数字
\end{lstlisting}

\subsubsection{字体、字号}
\begin{itemize}
\item[*] 正文字体默认为宋体,字体修改请修改文件\ttfamily{oucart.cls} \songti  \rmfamily 文件中的:
\begin{lstlisting}
\setCJKmainfont[BoldFont=STHeiti,ItalicFont=STKaiti]{STSong}
\end{lstlisting}
\item[*] 正文字号默认为小四,字号修改请更改以下代码中的\ttfamily{zihao} \songti \rmfamily 参数:
\begin{lstlisting}
\documentclass[zihao = -4,cn]{oucart} 
\end{lstlisting}
\end{itemize}
如需进行局部字号字体调整,可参考以下写法:
\begin{lstlisting}
\zihao{3} \heiti 三号黑体 \\
\zihao{4} \songti 四号宋体 \\
\zihao{5} \kaishu 五号楷书 \\
\end{lstlisting}
实现效果如下:\\
\zihao{3} \heiti 三号黑体 \\
\zihao{4} \songti 四号宋体 \\
\zihao{5} \kaishu 五号楷书 \\
\songti \zihao{-4}
\subsubsection{行距}
由于 \rmfamily \TeX{}行距计算方法与Microsoft Word存在差异,本模版中行距设置约相当于Microsoft Word中1.25倍行距。如需修改请更改\ttfamily{oucart.cls} \songti 文件中的:
\begin{lstlisting}
\linespread{1.35}
\end{lstlisting}
局部行距设置需要\ttfamily{setspace} \songti 宏包,示例用法为:
\begin{lstlisting}
\begin{spacing}{1.0}
\noindent 单倍行距 \\
单倍行距
\end{spacing}
\begin{spacing}{2.0}
\noindent 双倍行距 \\
双倍行距
\end{spacing}
\end{lstlisting}
实现效果如下:\\
\begin{spacing}{1.0}
\noindent 单倍行距 \\
单倍行距
\end{spacing}
\begin{spacing}{2.0}
\noindent 双倍行距 \\
双倍行距
\end{spacing}

\subsubsection{标题样式}
默认标题分为三级,它们与字体字号有以下对应关系:
\begin{itemize}
\item[*] 一级标题为黑体三号;
\item[*] 二级标题为黑体四号;
\item[*] 三级标题为黑体四号。
\end{itemize}
标题以阿拉伯数字编号,其基本样式为:\\


\noindent \zihao{3} \heiti{1. 一级标题} \\
\zihao{4} \heiti{1.1 二级标题} \\
\zihao{4} \heiti{1.1.1 三级标题} \\
\zihao{-4} \songti 如需修改,请修改\ttfamily{oucart.cls} \rmfamily \songti 文件中如下代码:
\begin{lstlisting}
\titleformat{\section}
            {\zihao{3}\heiti\raggedright}
            {\arabic{section}.}{1em}
            {}
\titleformat{\subsection}
            {\zihao{4}\heiti\raggedright}
            {\arabic{section}.\arabic{subsection}}{1em}
            {}
\titleformat{\subsubsection}
            {\zihao{4}\heiti\raggedright}
            {\arabic{section}.\arabic{subsection}.\arabic{subsubsection}}{1em}
            {}
\end{lstlisting}

\subsubsection{参考文献样式}
模版使用\BibTeX 管理参考文献,样式由\ttfamily{gbt7714.sty} \rmfamily \songti 指定,符合GB/T7714《文后参考文献著录规则》,样式示例\cite{wiki:ouc}。此样式来源于开源项目\href{https://github.com/zepinglee/gbt7714-bibtex-style}{GB/T 7714-2015 BibTeX Style}。\\
若期望使用其他样式,请删去\ttfamily{oucart.sty} \rmfamily \songti 文件中的以下代码:
\begin{lstlisting}
\RequirePackage{assets/gbt7714}
\end{lstlisting}
并在\ttfamily{.tex} \rmfamily \songti 文件中指定参考文献样式,例如:
\begin{lstlisting}
\bibliographystyle{unsrt}
\end{lstlisting}

\subsection{部分格式参考}
\subsubsection{图}
模版统一设置的图、表标题字号为五号,字体为黑体。建议使用的图片格式为\ttfamily{.eps/.pdf} \rmfamily \songti ,以下是一些例子:\\
\textbf{单张图片:}\\
代码:
\begin{lstlisting}
\begin{figure}[!htbp]
    \centering
    \includegraphics[width = 0.2\textwidth]{assets/logo}
    \caption{中国海洋大学}
    \label{fig:ouc1}
\end{figure}
\end{lstlisting}
实现效果:
\begin{figure}[!htbp]
    \centering
    \includegraphics[width = 0.2\textwidth]{assets/logo}
    \caption{中国海洋大学}
    \label{fig:ouc1}
\end{figure}

\noindent \textbf{两张图片:}\\
代码:
\begin{lstlisting}
\begin{figure}[hbp!] 
\centering 
\subfigure[\zihao{5}\heiti 海大]{ 
    \label{fig:subfig:a} 
    \includegraphics[width = 0.2\textwidth]{assets/logo}}
    \hspace{0.1in} 
\subfigure[\zihao{5}\heiti 海大]{ 
    \label{fig:subfig:b} 
    \includegraphics[width = 0.2\textwidth]{assets/logo}} 
\caption{中国海洋大学} 
\end{figure}
\end{lstlisting}
实现效果:
\begin{figure}[hbp!] 
\centering 
\subfigure[\zihao{5}\heiti 海大]{ 
    \label{fig:subfig:a} 
    \includegraphics[width = 0.3\textwidth]{assets/logo}}
    \hspace{0.1in} 
\subfigure[\zihao{5}\heiti 海大]{ 
    \label{fig:subfig:b} 
    \includegraphics[width = 0.3\textwidth]{assets/logo}} 
\caption{中国海洋大学} 
\end{figure}
\newpage
\noindent \textbf{三张图片:}\\
代码:
\begin{lstlisting}
\begin{figure}[!h]
        \subfigure[\zihao{5}\heiti 海大]{
        \includegraphics[width=0.3\textwidth]{assets/logo}}
        \hspace{0.1in} 
        \subfigure[\zihao{5}\heiti 海大]{
        \includegraphics[width=0.3\textwidth]{assets/logo}}
         \subfigure[\zihao{5}\heiti 海大]{
         \begin{minipage}{1\textwidth}
        \centering
        \includegraphics[width=0.3\textwidth]{assets/logo}
	\end{minipage}}
        \caption{中国海洋大学}
\end{figure}
\end{lstlisting}
实现效果:
\begin{figure}[!h]
        \subfigure[\zihao{5}\heiti 海大]{
        \includegraphics[width=0.4\textwidth]{assets/logo}}
        \hspace{0.1in} 
        \subfigure[\zihao{5}\heiti 海大]{
        \includegraphics[width=0.4\textwidth]{assets/logo}}
         \subfigure[\zihao{5}\heiti 海大]{
         \begin{minipage}{1\textwidth}
        \centering
        \includegraphics[width=0.4\textwidth]{assets/logo}
	\end{minipage}}
        \caption{中国海洋大学}
\end{figure}

\noindent \textbf{四张图片:}\\
代码:
\begin{lstlisting}
\begin{figure}[!hbp] 
\centering
	\subfigure[\zihao{5}\heiti 海大]{
	\includegraphics[width = 0.4\textwidth]{assets/logo}
	}\hfil
	\subfigure[\zihao{5}\heiti 海大]{
	\includegraphics[width = 0.4\textwidth]{assets/logo}
	}
	\subfigure[\zihao{5}\heiti 海大]{
	\includegraphics[width = 0.4\textwidth]{assets/logo}
	}\hfil
	\subfigure[\zihao{5}\heiti 海大]{
	\includegraphics[width = 0.4\textwidth]{assets/logo}
	}
	\caption{中国海洋大学}
\end{figure}
\end{lstlisting}
实现效果:
\begin{figure}[!hbp] 
\centering
	\subfigure[\zihao{5}\heiti 海大]{
	\includegraphics[width = 0.4\textwidth]{assets/logo}
	}\hfil
	\subfigure[\zihao{5}\heiti 海大]{
	\includegraphics[width = 0.4\textwidth]{assets/logo}
	}
	\subfigure[\zihao{5}\heiti 海大]{
	\includegraphics[width = 0.4\textwidth]{assets/logo}
	}\hfil
	\subfigure[\zihao{5}\heiti 海大]{
	\includegraphics[width = 0.4\textwidth]{assets/logo}
	}
	\caption{中国海洋大学}
\end{figure}

\subsubsection{表}
一个基本的三线表可以用以下代码实现:
\begin{lstlisting}
\begin{table}[!htbp]
\centering
\caption{一个基本的三线表}
\begin{minipage}[t]{350pt}
\begin{tabular*}{350pt}{@{\extracolsep{\fill}}ccc}
\toprule
第一列 & 第二列 & 第三列 \\
\midrule
文字 & English & $\alpha^*$ \\
文字 & English & $\beta$ \\
文字 & English & $\gamma$\\
\bottomrule
\end{tabular*}
\footnotesize
数据来源:相关的数据来源。 \\
*:表中需要解释的内容
\end{minipage}
\end{table}
\end{lstlisting}
实现效果:
\begin{table}[!htbp]
\centering
\caption{一个基本的三线表}
\begin{minipage}[t]{350pt}
\begin{tabular*}{350pt}{@{\extracolsep{\fill}}ccc}
\toprule
第一列 & 第二列 & 第三列 \\
\midrule
文字 & English & $\alpha^*$ \\
文字 & English & $\beta$ \\
文字 & English & $\gamma$\\
\bottomrule
\end{tabular*}
\footnotesize
数据来源:相关的数据来源。 \\
$*$:表中需要解释的内容
\end{minipage}
\end{table}

\subsubsection{数学相关}
\noindent 定理、引理等样式 \\
将以下代码加入到导言区:
\begin{lstlisting}
\newtheorem{theorem}{{定理}}[section]
\newtheorem{proposition}{{命题}}[section]
\newtheorem{lemma}{{引理}} [section]
\newtheorem{corollary}{{推论}}[section]
\newtheorem{definition}{{定义}} [section]
\newtheorem{example}{{例}} [section]
\end{lstlisting}
使用代码实例:
\begin{lstlisting}
\begin{theorem} 这是定理。\end{theorem}
\begin{proposition} 这是命题。\end{proposition}
\begin{lemma} 这是引理。\end{lemma}
\begin{corollary}这是推论。 \end{corollary}
\begin{definition} 这是定义。\end{definition}
\end{lstlisting}
效果如下:
\begin{theorem} 这是定理。\end{theorem}
\begin{proposition} 这是命题。\end{proposition}
\begin{lemma} 这是引理。\end{lemma}
\begin{corollary}这是推论。 \end{corollary}
\begin{definition} 这是定义。\end{definition}

公式:
\begin{equation}
 \lim_{x\to 0}{\frac{e^x-1}{2x}}
 \overset{\left[\frac{0}{0}\right]}{\underset{\mathrm{H}}{=}}
 \lim_{x\to 0}{\frac{e^x}{2}}={\frac{1}{2}}
\end{equation}
\section{修改记录}
\begin{itemize}
\item May,20 2015 \LaTeX{}模版v0.1;
\item May,20 2018 \LaTeX{}模版v0.2,\LaTeX{}使用手册第一版。
\end{itemize}

\newpage
\bibliography{main}

\end{document}
